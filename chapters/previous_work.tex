Two comprehensive reviews on electricity consumption forecasting have been identified:

\begin{itemize}
    \item An extensive review and comparison of both statistical and ML/DL techniques for forecasting is found in \cite{deb_review_2017}. They also review combinations of different techniques, i.e. hybrid models. Claims that ANN has more advantages than statistical models, and has better performance for nonlinear problems. Highlights that hybrid models can be beneficial to capture complexities in building energy and operational data.
    \item Another review of statistical, AI, and hybrid methods for forecasting is found in \cite{mat_daut_building_2017}. They also highlight the strength in AI models for dealing with nonlinear patterns. Claims that hybrids between AI and Swarm Intelligence (SI) methods show potential for increased accuracy. Provides a clear overview of different studies regarding prediction time intervals, included features, building types etc.
\end{itemize}

\noindent
Papers where experiments have been performed:

\begin{itemize}
    \item Support Vector Machine (SVM) for forecasting energy consumption: \cite{dong_applying_2005}.
    \item Monthly electricity consumption forecasting based on decomposition methods and ARIMA: \cite{sun_monthly_2019}.
    \item Forecasting cooling energy using ANN (for three university buildings, weekly \slash monthly): \cite{deb_forecasting_2016}.
    \item Forecasting high voltage consumers' electricity consumption using LSTM, GRU, TCN: \cite{hadjout_electricity_2022}.
    \item SVR and fruit fly optimization with seasonal indexing to address the fact that electricity consumption has a seasonal component: \cite{cao_support_2016}. Results show that the proposed model is a reliable forecasting tool.
\end{itemize}

\noindent
Papers specifically on XAI and energy consumption forecasting:

\begin{itemize}
    \item Forecasted hourly energy consumption for the steel sector using three different LSTM models \cite{maarif_energy_2023}. Used SHAP to interpret the decision-making, and found that leading current reactive power and the number of seconds from midnight contributed significantly to the model output.
    \item Ensemble learning for electricity consumption forecasting. Evaluated several decision tree-based ensemble learning techniques using SHAP \cite{moon_advancing_2024}. Found that temperature\hyp{}humidity index and wind chill temperature has a greater impact on short-term forecasts than more traditional parameters such as temperature. They also released the code\footnote{\url{https://github.com/sodayeong/PLOS-ONE_Github}}.
    \item Predicted electricity consumption for residential buildings based on hourly data with information about consumption for different household areas (such as kitchen and appliances) \cite{janjua_enhancing_2024}. Used LSTM as prediction model, and LIME and SHAP to provide comprehensible explanations of the predictions. 
    \item Proposed a methodology for selecting input variables for energy consumption prediction using XAI (SHAP) \cite{sim_explainable_2022}. Used Extreme Gradient Boosting (XGBoost), Support Vector Regression (SVR), Light Gradient Boosting Model (LightGBM), and LSTM for prediction. Found that variables with strong impact on the forecast include year, hour, energy consumption difference, temperature, and surface-temperature.
\end{itemize}

\section{Statistical Approaches}

\section{Machine Learning Approaches}
