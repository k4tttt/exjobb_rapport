% kanske kolla om detta går att ändra till european 
In recent years, the Swedish electricity market has been affected by large variations in price and periodically high electricity costs\footnote{Allhorn, J., Tidningen Näringslivet: Chockvändningen: Nu har norra Sverige högst elpriser \hyp{} ``Har förändrats ganska snabbt'' (January 2026). \url{https://www.tn.se/inrikes/46369/chockvandningen-nu-har-norra-sverige-hogst-elpriser-har-forandrats-ganska-snabbt/}}\footnote{Rydegran, E., Energiföretagen: Dramatik och rekord sammanfattar Elåret 2022 (December 2022). \url{https://www.energiforetagen.se/pressrum/pressmeddelanden/2022/Dramatik-och-rekord-sammanfattar-Elaret-2022/}}. This has increased the need for customers to understand and influence their electricity consumption and future electricity costs.

With the emergence of environmental problems due to climate change, sustainable energy management is becoming increasingly important \cite{chu_opportunities_2012,sim_explainable_2022}. The ability for customers to understand and influence their household energy consumption is significant for sustainable energy use. Household energy consumption makes up a considerable part of the total energy consumption, approximately 30\% in some European and American countries. Managing household energy more efficiently has large potential in saving energy, since it is estimated that 27\% of energy used by households can be saved through more efficient use \cite{zhou_understanding_2016}.

[To address this need, Umeå Energi has worked on developing and improving their interfaces for customers, for example by developing an application for private customers. According to customer analytics from Umeå Energi, a highly requested feature is to receive an estimation of both the electricity consumption and the electricity cost before the invoice arrives. Currently, Umeå Energi has developed a linear regression model to forecast the consumption for the current month. The model is based on historical consumption and temperature, aggregated by month. For certain customer groups, Umeå Energi has access to high definition measurement data where customer electricity consumption is registered on a 15-minute level, which creates opportunities for more advanced modeling. There may also be other factors, apart from historical consumption, that affect the electricity consumption, for example weather data, calendar information (holidays etc.), and energy prices.]

[Something something, forecasting electricity consumption can help users make smart decisions about their consumption, both saving money and benefitting the environment.]

For users, the ability to have confidence in the forecast is important. Users may, for example, question how the forecasting mechanism makes decisions, or question what factor is the most determining in the forecast. A system that can explain such things to the user is able to increase its reliability with users. Handling users' data in a safe and transparent manner is also important when implementing Artificial Intelligence (AI) based systems. Citizens of the European Union have a right to transparency and information about the decision-making of AI models that have a direct link to them, as outlined in the General Data Protection Regulation \cite{janjua_enhancing_2024}. [Something about how XAI helps with these things]

Previous research has identified various models for predicting electricity consumption, both statistical and AI based \cite{mat_daut_building_2017,deb_review_2017}. In recent years, increasing interest has been shown for AI solutions to energy consumption forecasting, due to their strength in handling nonlinear problems \cite{janjua_enhancing_2024,mat_daut_building_2017}. Forecasts are often based on several factors such as weather conditions and building conditions, causing nonlinear patterns that statistical methods struggle with \cite{mat_daut_building_2017}. However, one disadvantage with AI methods is the black-box nature of the models which causes a lack of transparency and interpretability in the decision-making process. To solve the problem of elucidating the internal process and decision-making of AI algorithms, the field of explainable AI (XAI) has emerged within recent years. XAI methods are able to open up the black box, and provide explanations for why a model predicts a certain way \cite{minh_explainable_2022}.

[The main goal with this Master's thesis is to investigate and quantify the contributions of different variables to a monthly electricity consumption forecast. Previous research has identified several models that are able to accurately predict energy consumption for different time intervals, utilizing various combinations of data sources \cite{deb_review_2017, mat_daut_building_2017}. In recent years, increasing interest has been shown for Machine Learning (ML) and AI solutions to energy consumption forecasting, due to their strength in handling nonlinear problems \cite{janjua_enhancing_2024,mat_daut_building_2017}. However, the black-box property of many popular AI models has increased the demand of asserting confidence and transparency through means of Explainable Artificial Intelligence (XAI) \cite{janjua_enhancing_2024}. This study will address this need by using XAI methods (SHAP, Feature Importance etc.) to evaluate how different variables such as weather data, holidays, and energy prices affect the accuracy of monthly household electricity consumption forecasts. The novelty in this research lies in explaining the variable contributions in the monthly scope. The access to customer-level data in this study also provides an aspect that is underrepresented in previous works \cite{mat_daut_building_2017,lazzari_user_2022}. In addition to the XAI approach, efforts will be made to continuously update and improve the forecast during the month, as actual observed values of electricity consumption become available.]

For Umeå Energi, this research will add value by supporting customers in making informed decision about their energy consumption. According to internal customer analytics, a highly requested feature is to receive an estimation of both the electricity consumption and the electricity cost before the invoice arrives. Umeå Energi is currently in the process of launching an application for private customers. The ultimate goal is to include to monthly consumption forecasts in the application.
%[Can explain the ultimate want from umeå energi, i.e. to show a reliable forecast in the app]